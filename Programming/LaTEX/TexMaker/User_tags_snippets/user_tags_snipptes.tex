% Author : Bhishan Poudel
% Date   : Apr 19, 2016
% Ref    : https://www.youtube.com/watch?v=UxCx78bgCTY
% Ref    : http://www.xm1math.net/texmaker/doc.html#SECTION33

\documentclass[10 pt, a4paper] {article}
\usepackage[utf8]{inputenc}
\usepackage{amsmath,amssymb,amsfonts}
\usepackage{graphicx,subfigure}
\usepackage[usenames,dvipsnames,svgnames,table]{xcolor}
\usepackage{hyperref}

\title{User tags in Texmaker}
\author{Bhishan Poudel}
\date{\today}
\author{Bhishan Poudel}
\begin{document}

\maketitle{}
%\tableofcontents
%\listoffigures
To make user defined tags in Texmaker.
\begin{verbatim}
Click on User icon on left panel (looks like man)\\
Right click on the empty space.\\
Item: parenth
Latex Content: $ \left( @ \right) $ 
Keyboard Trigger: par
click ok

Then, we can see user command parenth on the User
interface at left hand side.

To use: click it. or type   :par and press right arrrow  
\end{verbatim}
$ \left( x +1 \right) $ 


\begin{verbatim}
4.4 Personals tags ans tools

Texmaker allows you to insert your own tags (shortcuts : Shift+F1...Shift+F10). 
These tags are defined with the "User - User Tags" menu.
Notes :

    If the code of the menu is "%environment", Texmaker will directly insert:
    \begin{environment }
    •
    \end{environment }
    and the cursor will jump directly to the "•" field.
    You can also define an unlimited number of tags via the "User" panel in the "structure view" :
     just right-click on this panel to add or remove an item.
    All theses tags can be included in the document by clicking on an item in the panel or directly
     with the keyboard trigger ":foo" + key right.
    All the "@" characters in the code are automatically replaced by "•" place holders in the
     editor and the cursor will jump directly to the first "•" field (if some text has been
      selected in the editor before, the first "•" field will be replaced automatically by the
       selected text).
    Note : @@ will be replaced by the @ character in the editor (and not by a double placeholder)
    

You can also launch your own commands (shortcuts : Alt+Shift+F1...Alt+Shift+F5). These commands are
 defined with the "User - User Commands" menu and can be launched via the toolbar ("Run" button) .
 
\end{verbatim}

\end{document}