\documentclass[11pt,a4paper,english]{article}
\usepackage{babel}
\usepackage{amssymb}
\usepackage{graphicx,subfigure}
\usepackage[export]{adjustbox}    % for positioning figure
\usepackage{textcomp}
\usepackage{fixltx2e}
\usepackage[usenames,dvipsnames,svgnames,table]{xcolor}

%some useful newcommands
\newcommand{\beq}{\begin{equation}}
\newcommand{\eeq}{\end{equation}}
\newcommand{\bfig}{\begin{figure}}
\newcommand{\efig}{\end{figure}}
\newcommand{\beqa}{\begin{eqnarray}}
\newcommand{\eeqa}{\end{eqnarray}}
\newcommand{\beqan}{\begin{eqnarray*}}
\newcommand{\eeqan}{\end{eqnarray*}}
\newcommand{\ba}{\begin{array}}
\newcommand{\ea}{\end{array}}
\newcommand{\ben}{\begin{enumerate}}
\newcommand{\een}{\end{enumerate}}
\newcommand{\bfl}{\begin{flushleft}}
\newcommand{\efl}{\end{flushleft}}
\newcommand{\btab}{\begin{tabular}}
\newcommand{\etab}{\end{tabular}}
\newcommand{\bit}{\begin{itemize}}
\newcommand{\eit}{\end{itemize}}
\newcommand{\bdes}{\begin{description}}
\newcommand{\edes}{\end{description}}
\newcommand{\bdm}{\begin{displaymath}}
\newcommand{\edm}{\end{displaymath}}
\newcommand {\IR} [1]{\textcolor{red}{#1}}

% for listing
\usepackage{enumitem}
\usepackage[ampersand]{easylist}
\ListProperties(Hide=100, Hang=true, Progressive=3ex, Style*=-- ,
Style2*=$\bullet$ ,Style3*=$\circ$ ,Style4*=\tiny$\blacksquare$ )    % for easylist

% for hyperlink
\usepackage{hyperref}             % for hyperlink
\hypersetup{
    colorlinks=true,
    linkcolor=blue,
    filecolor=magenta,      
    urlcolor=cyan,
    pdftitle={Sharelatex Example},
    bookmarks=true,
    pdfpagemode=FullScreen,
}


% Creating Title for the assesment

\title{Assignment 2}
\author{Bhishan Poudel}
\date{Sep, 2015}
% Don't forget to use \maketitle below \begin{document}

\begin{document}
\maketitle
\tableofcontents
\listoffigures
\clearpage

\section{Care in evaluating functions}

	This program computes the equation
	\beqa
	y=x-sin x
	\eeqa
	for a wide range of values of x. \\

	\begin{easylist}
	& 	With a careful analysis of this function for values of x near zero,
		we used Taylor expansion of sine series near x equals 0.
	
	& 	From the `loss of precision theorem' we found that 
		for $|x| < 1.9$ we need to use Taylor expansion and beyond 
		that range we can use the general expression $x − sin x $,so, we did like this.
	
	& 	For $ |x| < 1.9 $, we used recurrence relation for 
		the terms in the series expansion in order to avoid
		having to compute very large factorials.
	& 	Here, we used single precision without Taylor expansion x near 0, 
		and plotted the graph.
		Then, we used double precision and used Taylor expansion for $|x| < 1.9$.\\
		The compiler is bright enough that we do not see much difference between two graphs.
	&	The source code and outputs are following:\\
		for single precision  : /assign2/qn1/ass2qn1Single.f90  and ass2qn1Single.dat\\
		for double precision  : /assign2/qn1/ass2qn1Double.f90  and ass2qn1Double.dat 
	\end{easylist}

	%%%%% including figure %%%%%%%%%%%%%%%%%%
	\begin{figure}
	\centering
	\includegraphics[width=1.0\textwidth,left]{./images/ass2qn1Single.eps}
	\caption{Plot of x vs $x - sin(x)$ }
	\end{figure}
	
	
	\begin{figure}
	
	\includegraphics[width=1.0\textwidth,left]{./images/ass2qn1Double.eps}
	\caption{Plot of x vs $x - sin(x)$ }
	\end{figure}
	%%%%%%%%%%%%%%%%%%%%%%%%%%%%%%%%%%%%%%%%%%
	
	
\clearpage
\section{Continued Fractions}

	Generally, when we encounter singular behaviors we rewrite 
	the function in terms of a Taylor expansion. 
	Another possibility is to use so-called continued fractions, which may be
	viewed as generalizations of a Taylor expansion. When dealing with continued fractions, 
	one possible approach is that of successive substitutions. 
	Let us illustrate this by this second order equation:	

	\beqa 
		x^2 + 4x -1 &=& 0 \\
        x    &=& 1/ (4 + x) \\  %gives first  +ve root x1=  0.23607
        x    &=& -4 + 1/x    %gives second -ve root x2= -4.23607
     \eeqa
     Here, second equation gives positive root and third equation gives negative root.\\ 
 	note: 4th iteration gives correct answer upto 5 digits after decimal\\
 	The source code and result are saved in the path:\\
 	assign2/qn2/ass2qn2.f90  and ass2qn2.dat\\
 	
\clearpage
\section{Bessel-functions via Recursion}

	In this question we calculated Bessel functions using upward and downward recursion relations.
	

	\subsection{• part a}
		In this part we calculated $j_l(x)$ values for first 25 l values for x = 0.1,1.0,and 10.0.
		First we used single precision and this gave floating point exception beyond l=20.
		The code fragment is below:
		
		\begin{verbatim}
			do l = 1,19 !eg. l = 1 (when we reach l = 20, we get floating point exception)
        
        			jup(l+1) = ( (2.d0*real(l)+ 1.d0)/x ) * jup(l)  - jup(l-1)
        			write(kwrite,*) x,l+1,jup(l+1)
        		end do 
		\end{verbatim}
		
		Then we used double precision. This time we did not get floating point exception.
		There were two options: upward and downward recursion; I chose downward recursion and normalize
		it in the end with known value of $j_0(x)$.
		The source code and results can be found here:
		single precision: assign2/qn3/ass2qn3Single.f90\\
		double precision: assign2/qn3/ass2qn3Double.f90\\
		
		results for x = 0.1,1.0,and 10.0 are respectively at:\\
		assign2/qn3/jdown\_A.dat \\
		assign2/qn3/jdown\_B.dat \\
		assign2/qn3/jdown\_C.dat \\
	
	
	\subsection{• part b}
		I tried both upward and downward recursion for spherical Bessel function.
		In the upward recursion relation with known value of $j_0(x)$ and $j_1(x)$,
		I calculated values upto $j_25(x)$ for x = 0.1.
		The true values for x = 0.1 can be found on Abramovitz Stegun table.
		I used double precision and took x = 0.1. For larger values of l, I found that
		downward recursion is better than upward recursion.
		I prepared a table like this:\\
		x l  jtrue(x)   jup(x)    jdown(x)    rd\_for\_up     rd\_for\_dn\\
		
		here, jtrue(x=0.1) values are taken from Stegun book table.\\
		rd means relative difference.\\
		we can see the table in:      \\
		assign2/qn3/rd\_wrt\_true\_A.dat  \\
		
		The source code is:\\
		assign2/qn3/ass2qn3Double.f90\\
		
	
	\subsection{• part c}
		In this part for downward recursion with large starting value of l, the
		fragment of source code looks like this:\\
		\begin{verbatim}
			jdown(51) = 2.d0 ! this value doesnot change ( if it is >= 19, we get floating point exception)
    			jdown(50) = 1.d0 ! this value doesnot change (Note: take diff values)

    			do l = 50, 1,-1 !eg. l=50
        			jdown(l-1) = ((2.d0*real(l)+1.d0)/x)*jdown(l)  - jdown(l+1) ! we get l-1 = 49
        			write(2,130)x,l-1,jdown(l-1)
          
    			end do
		\end{verbatim}
		The source code and values for l = 0 to 25 for x = 0.1,1.0,10.0 respectively can be found in:\\
		assign2/qn3/ass2qn3Double.f90\\
		assign2/qn3/jdown\_A.dat \\
		assign2/qn3/jdown\_B.dat \\
		assign2/qn3/jdown\_C.dat \\
		
		
	
	\subsection{• part d}
		In this part I compared upward and downward recursion methods. I formed a table looking like this:\\
		x     l       jup           jdown               relative difference \\
		The table can be found in:\\
		assign2/qn3/relative\_diff\_A.dat \\
		assign2/qn3/relative\_diff\_B.dat \\
		assign2/qn3/relative\_diff\_C.dat \\
		
		Here, A,B,C are for x =0.1,1.0,and,10.0 respectively.
		
	
	\subsection{• part e}
		Here, we found that the errors in the upward recursion depend on x.\\
		Looking the results we found that:\\
		when x = 0.1, jup and jdown values are approximately similar upto l = 4.\\
		when x = 1.0, jup and jdown values are approximately similar upto l = 8.\\
		when x = 10.0, jup and jdown values are approximately similar upto l = 25.\\
		
		The tables can be found in:\\
		assign2/qn3/relative\_diff\_A.dat \\
		assign2/qn3/relative\_diff\_B.dat \\
		assign2/qn3/relative\_diff\_C.dat \\
		
		Here, A,B,C are for x =0.1,1.0,and,10.0 respectively.
		This can be explained like this: when x increases the values we take for
		initial values of $j_0(x)$ decreases,i.e. $j_0(0.1) > j_0(1.0) > j_0(10.0) $.
		
		Now, the upward recursion relation is:\\
		\begin{equation}
		j_{l+1}(x) =  ((2l + 1)/x) j_l(x) - j_{l-1}(x) 
		\end{equation}
		
		Here we use recursion relation to calculate higher values of $j_l(x)$ by subtracting two terms
		as shown above.		
		
		There is larger error in subtracing small value from large number and smaller is the
		error when subtracting two similar numbers. Hence the result follows.



\end{document}

