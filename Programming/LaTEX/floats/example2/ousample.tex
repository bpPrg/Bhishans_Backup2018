% OUPROP.TEX -- electronic template for OU Observatory proposals
% Hacked up from the NOAO STANDARD proposal form from 1999

% THE FORM STARTS HERE
%

% Please do not modify or delete this line.
\documentstyle[nprop21,11pt]{article}

% Please do not modify or delete this line.
\begin{document}

Bhishan's note on compilation:\\
Using Texmaker\\
==============\\
1. Latex    F5 (my shortcut)   	creates dvi,log,aux files\\
   Pdflatex F6 (my shortcut)  	creates pdf without fig.\\
   viewpdf  F7 (my shortcut)  	opens   pdf without fig.\\
   
2. Latex    F5 (my shortcut)   	creates dvi,log,aux files\\
   dvi-pdf  F8 (my shortcut)		creates pdf with overlapped fig.\\
   viewpdf  F7 (my shortcut)		opens   pdf with overlapped fig.\\
   
Using Terminal\\
==============\\
Open in terminal ctrl shf T    	opens the terminal\\
latex 	ousample.tex           	creates dvi,log,aux files\\
dvipdf 	ousample.dvi            	creates pdf with no overlapped fig.\\
xdg-open ousample.pdf			opens   pdf with no overlapped fig.\\
note: my alias for xdg-open is get\\
\\
\\		

% Please do not modify or delete this line.  
\proposaltype{}

%%%%%%%%%%%%%%%%%%%%%%%%%%%%%%%%%%%%%%%%%%%%%%%%%%%%%%%%%%%%%%%%%%%%%%%%%%%

% SCIENTIFIC CATEGORIES
%
% Please select a "scientific category" that best describes your program by
% uncommenting (removing the % sign from) only ONE of the selections below. 
% DO NOT MODIFY THE SELECTION YOU UNCOMMENT.

%\sciencecategory{Cosmology}
%\sciencecategory{Clusters of Galaxies}
\sciencecategory{Galaxies}
%\sciencecategory{Star Clusters}
%\sciencecategory{Interstellar Medium}
%\sciencecategory{Stars}
%\sciencecategory{Planets}
%\sciencecategory{Small Bodies \& Moons}
%\sciencecategory{Other}

%%%%%%%%%%%%%%%%%%%%%%%%%%%%%%%%%%%%%%%%%%%%%%%%%%%%%%%%%%%%%%%%%%%%%%%%%%%

% TITLE
%
% Give a descriptive title for the proposal in the \title command.
%
% Note that a title can be quite long; LaTeX will break the title into
% separate lines automatically.  If you wish to indicate line breaks
% yourself, do so with a `\\' command at the appropriate point in
% the title text.  Use both upper and lower case letters (NOT ALL CAPS).

\title{The Surface Brightness Profile of the\\ Elliptical Galaxy NGC 3379}

%%%%%%%%%%%%%%%%%%%%%%%%%%%%%%%%%%%%%%%%%%%%%%%%%%%%%%%%%%%%%%%%%%%%%%%%%%%

% INVESTIGATOR'S (PI AND CoI) INFORMATION BLOCKS
%
% Please give the PI's name, email address, and local phone number.
% Also indicate the principal investigator's status with 
% one of the one-letter codes inside the \invstatus{} curly braces, as 
% indicated below.
%
% If the proposal involves a collaboration, uncomment the co-I block and
% give the same information for co-I (collaborator).
% REMEMBER THAT COLLABORATIVE EFFORTS MUST BE APPROVED IN ADVANCE.
%
% Status codes:
%
%    \invstatus{P}  % investigator has obtained PhD or doctorate
%    \invstatus{T}  % investigator is grad student, observing for thesis
%    \invstatus{G}  % investigator is grad student, not observing for thesis
%    \invstatus{U}  % investigator is an undergraduate student
%    \invstatus{O}  % investigator has other status (none of the above)
%
%
% DO NOT remove the \begin{PI} and \end{PI}.  Only one individual's name per 
% \name field is allowed.

\begin{PI}
\name{Jehosophat Y. Floppybottom}
\email{floppybottom@helios.phy.ohiou.edu}
\phone{1-800-MOO-OINK}
\invstatus{U}
\end{PI}

%\begin{CoI}
%\name{}
%\email{}
%\phone{}
%\invstatus}
%\end{CoI}

%%%%%%%%%%%%%%%%%%%%%%%%%%%%%%%%%%%%%%%%%%%%%%%%%%%%%%%%%%%%%%%%%%%%%%%%%%%%

% ABSTRACT
%
% Give a general abstract of the scientific justification appropriate for
% a non-specialist.  Write the abstract between the \begin{abstract} and
% \end{abstract} lines.  Limit yourself to approximately 175 words.

% DO NOT remove the \begin{abstract} and \end{abstract} lines.

\begin{abstract}
NGC 3379 is regarded in the astronomical literature as a near-perfect
illustration of the famous $r^{1/4}$ law, introduced by G.\ de Vaucouleurs
to describe the radial dependence of the surface brightness in elliptical
galaxies. The goal of this project is to independently reproduce this
result and confirm that the de Vaucouleurs law accurately describes this
galaxy, through deep imaging in the $R$ band. In addition this project will
test the ability of the current hardware to perform accurately at low surface
brightness levels.
\end{abstract}

%%%%%%%%%%%%%%%%%%%%%%%%%%%%%%%%%%%%%%%%%%%%%%%%%%%%%%%%%%%%%%%%%%%%%%%%%%%%

% SUMMARY OF OBSERVING RUN
%
% List a summary of the details of the observing run.
%
%   \begin{obsrun}
%   \telescope{}        % Must be GOT (only telescope available)
%   \instrument{}       % Must be Direct CCD (only instrument available)
%   \numnights{}        % Total nights requested, e.g. \numnights{2}
%   \lunardays{}        % Maximum tolerable moon age, e.g. \lunardays{14}
%   \optimaldates{}     % E.g. \optimaldates{Apr 1-30}; explain in next section
%   \acceptabledates{}  % Must be Apr - May
%   \end{obsrun}
%
% \numnights should give the number of nights of the run.
% Formats such as 5x0.5 are acceptable if your targets are accessible only
% before or after midnight.
%
% \lunardays should specify the maximum number of nights from new moon 
% which can be utilized to accomplish your scientific goals.  It should 
% be a number from 0 (new moon) to 14 (full moon).
%

\begin{obsrun}
\telescope{GOT}			% Leave as is
\instrument{Direct CCD}		% Leave as is
\numnights{1}
\lunardays{14}
\optimaldates{any}
\acceptabledates{Apr - May}	% Leave as is
\end{obsrun}

% If there are scheduling constraints or non-usable dates
% (i.e., other than moon phase or when your object is up)
% give the dates by filling in the curly braces in \unusabledates{}.
%
% Please limit your text to 4 lines on the printed copy.

\unusabledates{}

%%%%%%%%%%%%%%%%%%%%%%%%%%%%%%%%%%%%%%%%%%%%%%%%%%%%%%%%%%%%%%%%%%%%%%%%%%%%

% In the following "essay question" sections, the delimiting pieces of
% markup (\justification, \expdesign, etc.) act as LaTeX \section*{}
% commands.  If the author wanted to have numbered subsections within
% any of these, LaTeX's \subsection could be used.
%
% DO NOT REDUCE THE FONT SIZE, and do not otherwise fiddle with the format 
% to get more on a page. 

% SCIENTIFIC JUSTIFICATION
%
% Give the scientific justification for the proposed observations.
% Be sure to explain the significance of your observations in the context
% of what has been done previously. This section should define the SCIENCE
% goals of the project, rather than the technical requirements, which should go
% in the next section.
% THE SCIENTIFIC JUSTIFICATION SHOULD BE LIMITED TO ONE PAGE
% plus one additional page for references, figures, and captions.

% For references, follow the example below
% (journal commands are compatible with AASTeX v4.0):
%
%\begin{references}
%\reference Armandroff \& Massey 1991 \aj, 102, 927.
%\reference Berkhuijsen \& Humphreys 1989 \aap, 214, 68.
%\reference Massey 1993 in Massive Stars: Their Lives in the Interstellar
%  Medium (Review), ed. J. P. Cassinelli and E. B. Churchwell, p. 168.
%\reference Massey \& Armandroff 1999, in prep.
%\end{references}

% Only encapsulated postscript figures may be included with your proposal.
% To include an EPS plot, use the LaTeX "figure" environment.  
% The plot file is included with the \plotone{FILENAME} command; two 
% side-by-side plot files can be included by typing
% \plottwo{FILENAME1}{FILENAME2}.  Use \caption{} to specify a caption.
% The \epsscale{} command can be used to scale \plotone plots if they
% appear too large on the printed page. For example:
%
% \begin{figure}
% \epsscale{0.85}
% \plotone{sample.eps}
% \caption{Sample figure showing important results.}
% \end{figure}
%
% If you need to rotate or make other transformations to a figure, you may 
% use the \plotfiddle command:
% \plotfiddle{PSFILE}{VSIZE}{ROTANG}{HSCALE}{VSCALE}{HTRANS}{VTRANS}
% \plotfiddle{sample.eps}{2.6in}{-90.}{32.}{32.}{-250}{225}
% where HSCALE and VSCALE are percentages and HTRANS and VTRANS are 
% in PostScript units, 72 PS units = 1 inch.

\sciencejustification

The giant elliptical galaxy NGC 3379 (M105) is one of the best-studied
galaxies in the sky. Determinations of its surface brightness profile date
back all the way to the work of Hubble (1930), who was able to measure the
azimuthally averaged radial profile extending over $\sim 1.5$ orders of
magnitude in radius, corresponding to $\sim 2$ orders of magnitude in
surface brightness.  Much later, in what is generally recognized as a
{\it tour de force\/} of photographic and photoelectric photometry,
de Vaucouleurs \& Capaccioli (1979) extended the measured profile to
nearly 4 orders of magnitude in surface brightness. They showed
convincingly that the observed profile was remarkably well fit by the
formula suggested by de Vaucouleurs (1948),
\begin{equation}\label{e.devlaw}
\Sigma(r) = \Sigma_e \exp \left\{-7.67\left[(r/r_e)^{1/4}-1\right]\right\}.
\end{equation}
In equation (\ref{e.devlaw}), $\Sigma(r)$ is the surface brightness at
radius $r$ on the sky, expressed in linear (physical) units, $r_e$ is the
``effective radius'' of the galaxy, and $\Sigma_e$ is the surface
brightness at $r=r_e$. For a galaxy that precisely obeys equation
(\ref{e.devlaw}), half its total light comes from $r<r_e$. 
The functional form is such that if one plots $\Sigma$ on a logarithmic
scale, e.g., as mag$\,$arcsec$^{-2}$, against $r^{1/4}$, one obtains a
straight line. The data for NGC 3379, plotted in this way, are shown in
Figure 1 (Fig.\ 2 of de Vaucouleurs \& Capaccioli 1979).

Even though subsequent work has shown that equation (\ref{e.devlaw})
does not fit all elliptical galaxies at all radii, it has proved sufficiently
useful that it is now commonly referred to as the $r^{1/4}$ {\it law\/}
or the {\it de Vaucouleurs law.} The primary goal of this project is, in
effect, to rediscover the $r^{1/4}$ law using CCD imaging of NGC 3379
with the GOT.

A number of more recent CCD studies of NGC 3379 have been published,
(e.g., Peletier et al.\ 1990), extending to radii $> 100\arcsec$,
showing small deviations from the $r^{1/4}$ law. Though the
interpretation of these deviations has been a matter of some debate
(Capaccioli et al.\ 1991, Hjorth \& Madsen 1991, Statler 1994), the
availability of high-quality data in the literature provides a means of
checking our results. In particular, we have an opportunity to test the
performance of the GOT at low surface brightness levels; this is the
secondary goal of the project. The Peletier et al.\ data indicate that
$\mu_R(200\arcsec) \approx 23.5\,{\rm mag}\,{\rm arcsec}^2$, or about
$6\%$ of the $R$ band sky brightness at new moon.

Potentially lying within the same CCD field as NGC 3379 are the SB0
galaxy NGC 3384 and the Sc galaxy NGC 3389. Their presence is both a
hindrance and a benefit. Because these galaxies will collectively fill
the frame, it will be necessary to observe a separate blank-sky field
to determine the sky surface brightness. On the other hand, we will also
be able to measure their surface brightness profiles and demonstrate the
difference between the profiles of E, S0, and S galaxies.

Finally, the entire NGC 3379 group is known to reside within a ring of
neutral hydrogen (Figure 2; Schneider 1985). While there
are no known optical signs of interaction between NGC 3379, NGC 3389, and
NGC 3384, our long exposures will, as a byproduct, produce a deep---and
presumably highly photogenic---image of this attractive three galaxy set.

% End of the justification main text. The following line
% cleans up the page, and goes to the next page for refs and figs.
\clearpage
% Now we're on the page for figures and references.

\begin{references}
\reference Capaccioli, M., Vietri, M., Held, E. V., \& Lorenz, H. 1991,
        ApJ, 371, 535
\reference de Vaucouleurs, G. 1948, Ann.\ d'Ap., 11, 247
\reference de Vaucouleurs, G. \& Capaccioli, M. 1979, ApJS, 40, 699
\reference Hjorth, J. \& Madsen, J. 1991, MNRAS, 253, 703
\reference Hubble, E. P. 1930, ApJ, 71, 231
\reference Peletier, R. F., Davies, R. L., Davis, L. E., Illingworth,
        G. D., \& Cawson, M. 1990, AJ, 100, 1091
\reference Schneider, S. E. 1985, ApJL, 288, L33
\reference Statler, T. S. 1994, AJ, 108, 111
\end{references}

\begin{figure}
\epsscale{.5}
\plottwo{fig1.eps}{fig2.eps}
\caption{(Left) Radial surface brightness profile of NGC 3379, reproduced
from de Vaucouleurs \& Capaccioli (1979). The $V$-band surface brightness
$\mu_V$ (in mag$\,$arcsec$^{-2}$) is plotted against the radius in arcsec to
the $1/4$ power. Points show the data (error bars omitted by the authors);
the line is the best fit $r^{1/4}$ law.}
\caption{(Right) Map of the intergalactic HI ring surrounding NGC 3379
(near the field center),
NGC 3389, and NGC 3384 (reproduced from Schneider 1985). Dotted outline shows
a characteristic 21 cm isophote; solid contours indicate radial
velocity. Field is approximately $1.5 \times 2$ degrees.}
\end{figure}

% End of the refs/figs page. Clean up and go on
\clearpage


% EXPERIMENTAL DESIGN AND TECHNICAL DESCRIPTION
%
% This section should consist of text only (no figures).
% There is no limit on length.

% Describe the observations to be made during this observing run.
% Include exposure time calculations to demonstrate that 
% these observations will accomplish the science goals outlined in the 
% previous section. Include estimates for calibration exposures.
% Justify the number of nights and the lunar phase requested. 
% List objects, coordinates, and magnitudes (or surface brightness, if 
% appropriate) in a Target Table in the next section.

\expdesign

We choose $R$ band as the best compromise between the general red color
of elliptical galaxies, the very red night sky, and the maximum sensitivity
of the detector. (In addition, more published $R$-band than $I$-band photometry
is available for NGC 3379.)

With NGC 3379 centered in the field we can hope to work to a maximum
radius of $360\arcsec$; extrapolating the photometry of Peletier et al.
indicates $\mu_R \approx 25$ (about $1.5\%$ of sky) at this distance. We
set an accuracy goal of $\pm 0.1$ mag ($S/N=10$) at this radius. We will be
fitting isophotes using the IRAF/STSDAS ``ellipse'' task, which, for each
isophote, takes data from a ring effectively 1 pixel wide. This implies,
with the $0.7\arcsec$/pixel plate scale (and assuming circular isophotes),
approximately 3200 pixels will be averaged. However, we expect to be
able to use only the western half of the galaxy at this radius because
of contamination from NGC 3384 and NGC 3389 to the east. We determine
the expected $S/N$ for an average over 1600 pixels using the ``ccdtime''
calculator, assuming a mean airmass of $1.4$ and a CCD temperature of
$-15^\circ$C. We obtain $S/N=3.8$ for a $300\,$s exposure and $S/N=6.4$
for a $600\,$s exposure. Thus we can reach $S/N=10$ in $7 \times 300\,$s
or $3 \times 600\,$s. The detector is the primary source of noise, so,
since we are working in $R$ band, this project (surprisingly) can be
done at any lunar phase.

However, the photometric accuracy will ultimately be limited not by
photon counting statistics but by systematic variation of the sky.
Since the sky is 70 times brighter than the galaxy at these radii, the
sky level must be determined to better than $0.5\%$. Even in photometric
conditions we can expect the sky to vary by more than this as the target
moves toward the horizon. Since the galaxies will fill the field we must
``beam switch'' to a nearby blank field to get sky; and because the sky
is varying in time, more rapid switching is advantageous. Therefore we
opt for $300\,$s exposures on the galaxies alternating with blank sky
exposures. The mean sky level will be determined by masking out stars
and averaging over the entire source-free part of the image, which we can
expect to comprise $N\approx 10^6$ pixels. The ``ccdtime'' task reports
that we should expect $\sim 0.43\,{\rm e}^-\,{\rm s}^{-1}\,{\rm pix}^{-1}$
from the sky at new moon. At an operating temperature of $-15^\circ$C, we
should have $0.08\,{\rm e}^-\,{\rm s}^{-1}\,{\rm pix}^{-1}$ dark current,
and the CCD read noise is $11.8\,{\rm e}^-$ RMS. Thus
\begin{equation}
\left({S\over N}\right)_{\rm sky}
	= {0.43 t N^{1/2} \over [0.43 t + 0.08 t + (11.8)^2]^{1/2}},
\end{equation}
where $t$ is the exposure time in seconds. For a $30\,$s exposure we
should achieve $(S/N)_{\rm sky}=1000$, or $0.1\%$ accuracy. These
exposures are short enough that they can be unguided, which will reduce
overhead substantially.

In early May NGC 3379 will be at airmasses $<2$ for at least 3 hours
after the end of evening twilight. The total exposure time required is
$7 \times 300\,{\rm s} + 8\times 30\,{\rm s} = 39\,$ min. If we allot
10 minutes of overhead for each exposure (CCD readout, moving the
telescope, reacquiring the guide star) the total time required sums to
189 minutes. We will therefore be able to acquire the needed data
in half a night, neglecting photometric calibration. We emphasize that
the primary science of this project will still be doable in non-photometric
conditions, but comparison with published data will be possible only to
within an undetermined photometric zero point. If we have photometric
conditions, then calibrating the photometry will require a minimum of
2 standard star observations. We will choose one or more standards with
colors similar to that of NGC 3379 and observe them at airmasses between
$1.1$ and $2.0$. If possible we will scatter the standards among the
galaxy exposures. However, if we are running behind schedule we will put
off the standards to the second half of the night in order not to
sacrifice on-source time, which is essential to the primary science.
We estimate standards to require $\sim 1$ hour total, which may be
interspersed between other observations. Beginning observations on NGC
3379 at the end of twilight will require extremely efficient use of the
90 minutes after sunset for flat fields, zeros, darks, polar alignment,
focusing, and autoguider calibration. This will be possible only if the
observers are very experienced with the equipment.

% TARGET TABLES
%
% List primary targets below. (Photometric standard star fields, flat
% fields, etc. do not need to be included here.) List the same object
% observed with three different filters as 3 objects; give exposure
% information separately for each filter.
%
% HINTS: Long tables do not break across pages. If it is necessary to 
% continue a table across a page you must start a new table.  Use 
% /clearpage before the \begin{targettable} command for the new table.  
%
%\begin{targettable}{}
%\objid{}           % Arbitrary ID, use consecutive numbers
%\object{}          % 20 characters maximum
%\ra{}              % e.g., xx:xx:xx.x
%\dec{}             % e.g., +-xx:xx:xx.x
%\epoch{}           % Equinox of the coordinate system, e.g., B1950 or J2000
%\magnitude{}	    % Total Magnitude or surface brightness at point of interest
%\filter{}	    % V, R, I, H$\alpha$, or H$c$ (continuum)
%\exptime{}         % seconds PER EXPOSURE
%\nexposures{}      % Number of exposures
%\moondays{}        % Days from new moon, use a number 0-14
%\skycond{}         % Leave blank
%\seeing{}          % Leave blank
%\obscomment{}      % Comment, 20 characters maximum
%  -- repeat target entry parameters as needed to complete Target Table --
%\end{targettable}

\begin{targettable}{GOT/Direct}

\objid{1}
\object{NGC 3379}
\ra{10 47 49.6}
\dec{+12 34 55 }
\epoch{2000.0}
\magnitude{10.3}
\exptime{300}
\nexposures{7}
\moondays{0-14}
\skycond{}
\seeing{}
\obscomment{mag is $B^T$}

\objid{2}
\object{Sky}
\ra{10 46 30.0}
\dec{+12 17 00 }
\epoch{2000.0}
\magnitude{N/A}
\exptime{30}
\nexposures{8}
\moondays{0-14}
\skycond{}
\seeing{}
\obscomment{Blank field $25^\prime$ SW}

\end{targettable}


%%%%%%%%%%%%%%%%%%%%%%%%%%%%%%%%%%%%%%%%%%%%%%%%%%%%%%%%%%%%%%%%%%%%%%%%%%%%

% Please do not modify or delete this line.
\end{document}

%%%%%%%%%%%%%%%%%%%%%%%%%%%%%%%%%%%%%%%%%%%%%%%%%%%%%%%%%%%%%%%%%%%%%%%%%%%%
