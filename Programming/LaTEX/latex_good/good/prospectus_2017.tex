\documentclass[12pt]{article}
\usepackage{graphicx, wrapfig, subcaption, setspace, booktabs}
\usepackage[T1]{fontenc}
\usepackage[english]{babel}
\usepackage{hyperref}
\usepackage{amsmath,amssymb}
\usepackage[numbers]{natbib}
\bibliographystyle{plainnat}
\graphicspath{ {images/} }


\begin{document}


\thispagestyle{empty}

%\vspace*{0.5cm}
\begin{center}
	{\fontsize{24pt}{32pt}\usefont{OT1}{cmr}{m}{}
		PSF Correction and Wavelength Dependence}
	
	\vspace{3cm}
	
	{%\fontsize{18pt}{18pt}
		\sc
		Ph.D. Thesis prospectus by Bhishan Poudel\\
		(Advisor: Prof. Douglas Clowe)}
	
	\vspace{2cm}
	
	{\it
		Department of Physics and Astronomy\\
		Ohio University\\
		Athens, OH 45701\\}
	\vspace{1cm}
	bp959314@ohio.edu
\end{center}

\vfill
\noindent Prof. Douglas Clowe (Department of Physics \& Astronomy)\\
Prof. Ryan Chornock (Department of Physics \& Astronomy)\\

\newpage
\tableofcontents
\clearpage
%\tableofcontents

\section{Introduction}


According to standard model of cosmology dark matter accounts for  28.5\% of total mass energy content of the Universe.  It's mainly dominated in Galaxy and Galaxy Clusters.  Anisotropy in the Cosmic microwave background,cosmic structure formation,galaxy formation
and evolution suggest the presence of dark matter.  As it does not interact with electromagnetic radiation and visible matter the method it can  be detected is by observing it's effect on ordinary baryonic matter.  Gravitational Lensing provides us a way
to see how dark matter along with visible matter is distributed in a galaxy or galaxy cluster.  This is supported by Einstein's general theory of relativity which predicts the deflection of light in a gravitational field produced by a massive object.

(refer to: \citet{soldner04})

\clearpage
\bibliography{bib/lensing}

\end{document}