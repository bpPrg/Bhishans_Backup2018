\documentclass[12pt]{article}
\usepackage{graphicx, wrapfig, subcaption, setspace, booktabs}
\usepackage[T1]{fontenc}
\usepackage[english]{babel}
\usepackage{hyperref}
\usepackage{amsmath,amssymb}
\usepackage[numbers]{natbib}
\bibliographystyle{plainnat}
\graphicspath{ {images/} }


\begin{document}





\thispagestyle{empty}

%\vspace*{0.5cm}
\begin{center}
        {\fontsize{24pt}{32pt}\usefont{OT1}{cmr}{m}{}
                PSF Correction and Wavelength Dependence}

        \vspace{3cm}

        {%\fontsize{18pt}{18pt}
                \sc
                PH.D. Thesis prospectus by Bhishan Poudel\\
                (Advisor: Prof. Douglas Clowe)}

        \vspace{2cm}

        {\it
                Department of Physics and Astronomy\\
                Ohio University\\
                Athens, OH 45701\\}
        \vspace{1cm}
        bp959314@ohio.edu
\end{center}

\vfill
\noindent Prof. Douglas Clowe (Department of Physics \& Astronomy)\\
Prof. Ryan Chornock (Department of Physics \& Astronomy)\\

\newpage
\tableofcontents
\clearpage
%\tableofcontents

\section{Introduction}


According to standard model of cosmology dark matter accounts for  28.5\% of total mass energy content of the Universe.  The dark matter is mainly dominated in Galaxy and Galaxy Clusters.  Anisotropy in the Cosmic microwave background, cosmic structure formation,  galaxy formation and evolution suggest the presence of dark matter.  As it does not interact with electromagnetic radiation and visible matter the method it can  be detected is by observing it's effect on ordinary baryonic matter.  Gravitational Lensing provides us a way to see how dark matter along with visible matter is distributed in a galaxy or galaxy cluster.  This is supported by Einstein's general theory of relativity which predicts the deflection of light in a gravitational field produced by a massive object.

(refer to: \citet{soldner04})
\clearpage
\section{Future Projects}
This is the section for future projects.
This is the section for future projects.

\clearpage
\section{Timeline}
I intend to complete all elements of this project within two years. An outline of the projected time line is presented in the table below. The dates and activities displayed are subject to change should
unforeseen and interesting developments occur during the projects that may require special attention and time.

\begin{table}[h]
    \begin{center}
        \begin{tabular}{|c|c|c|}
            \hline
            Start&Finish&Activity\\
            \hline
            Sep. 2016& March 2017 &  \\
            \hline
            April 2017 & Sep. 2017 &   \\
            \hline
            Sep. 2017 & March 2018 & Preparation of manuscripts for publication \\
            \hline
            March 2018 & Aug. 2018& Writing and defending the dissertation   \\
            \hline
        \end{tabular}
    \end{center}

\end{table}

\clearpage
\bibliographystyle{plain}
\bibliography{bib/lensing.bib}


\end{document}

\bibliography{bib/lensing}

\end{document}
