% ref  : https://www.sharelatex.com/learn/Aligning_equations_with_amsmath

\documentclass{article}
\usepackage{blindtext}
\usepackage[utf8]{inputenc}
\usepackage{amsmath}


\title{amsmath package}
\author{Bhishan Poudel}
\date{\today}

\begin{document}
\maketitle
\tableofcontents{}

\section{Section 1}

	You have to wrap you equation in the equation environment 
	if you want it to be numbered, use equation* 
	(with an asterisk) otherwise. Inside the equation environment 
	use the split environment to split the equations into smaller 
	pieces, these smaller pieces will be aligned accordingly. 
	The double backslash works as a newline character. 
	Use the ampersand character \&, to set the points 
	where the equations are vertically aligned.


\begin{equation*} \label{eq1}
\begin{split}
A &= \frac{\pi r^2}{2} \\
 &= \frac{1}{2} \pi r^2
\end{split}
\end{equation*}

\section{Writing a single equation}
\begin{equation} \label{eu_eqn}
e^{\pi i} - 1 = 0
\end{equation}
 
The beautiful equation \ref{eu_eqn} is known as the Euler equation.

\section{Displaying long equations}
\begin{multline*}
p(x) = 3x^6 + 14x^5y + 590x^4y^2 + 19x^3y^3\\ 
- 12x^2y^4 - 12xy^5 + 2y^6 - a^3b^3
\end{multline*}

\section{Aligning several equations}
\begin{align} 
2x - 5y &=  8 \\ 
3x + 9y &=  -12
\end{align}

\section{Aligning multiple equations}
\begin{align*}
x&=y           &  w &=z              &  a&=b+c\\
2x&=-y         &  3w&=\frac{1}{2}z   &  a&=b\\
-4 + 5x&=2+y   &  w+2&=-1+w          &  ab&=cb
\end{align*}

\section{Grouping and centering equations }
\begin{gather*} 
2x - 5y =  8 \\ 
3x^2 + 9y =  3a + c
\end{gather*}





\end{document}