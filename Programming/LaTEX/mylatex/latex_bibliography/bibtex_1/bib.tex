\documentclass[a4paper,10pt]{article}
\usepackage[english]{babel}
\usepackage[utf8]{inputenc}
\usepackage{hyperref}
\usepackage{url}

%Includes "References" in the table of contents
\usepackage[nottoc]{tocbibind}

%Title, date an author of the document
\title{Bibliography management: BibTeX}
\author{Share\LaTeX}

%Begining of the document
\begin{document}

\maketitle
\tableofcontents
\medskip

\section{First Section}
This document is an example of BibTeX using in bibliography management. 
Three items are cited: \textit{The \LaTeX\ Companion} book \cite{latexcompanion}, 
the Einstein journal paper \cite{einstein}, and the Donald Knuth's 
website \cite{knuthwebsite}. The \LaTeX\ related items are 
\cite{latexcompanion,knuthwebsite}. 

\section{Overview}
Before we get to the commands and facilities provided by Biblatex, we will have
a look at some of its fundamental concepts. The Biblatex package uses auxiliary
files in a special way. Most notably, the bbl
file is used differently and when using
BibTeX as the backend, there is only one
bst
file which implements a structured data
interface rather than exporting printable data. With LaTeX’s standard bibliographic
facilities, a document includes any number of citation commands in the document
body plus
\textbf{bibliographystyle}
and
\textbf{bibliography}
, usually towards the
end of the document. The location of the former is arbitrary, the latter marks the
spot where the list of references is to be printed

\newpage
\bibliographystyle{plain} % alpha,unsrt, IEEE,plain, nature, abbrv etc.
\bibliography{sample}
\end{document}