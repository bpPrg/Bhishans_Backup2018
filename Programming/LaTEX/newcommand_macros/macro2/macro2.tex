\documentclass{article} 
 
\begin{document}

Examples of macros in latex.\\



\newcommand{\plusbinomial}[3][2]{(#2 + #3)^#1}
 
To save some time when writing too many expressions 
with exponents is by defining a new command to make simpler:
 
\[ \plusbinomial{x}{y} \]
 
And even the exponent can be changed
 
\[ \plusbinomial[4]{y}{y} \]

\begin{verbatim}
Let's analyse the syntax of the line \newcommand{\plusbinomial}[3][2]{(#2 + #3)^#1}:

\plusbinomial
    This is the name of the new command. 

[3]
    The number of parameters the command will take, in this case 3. 

[2]
    Is the default value for the first parameter. This is what makes the first parameter optional, if not passed it will use this default value. 

(#2 + #3)^#1
    This is what the command does. In this case it will put the second and third parameters in a "binomial format" to the power represented by the first parameter. 
\end{verbatim}

\end{document}