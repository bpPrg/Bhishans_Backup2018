\documentclass{article}
\usepackage[utf8]{inputenc}

\usepackage{listings}
\usepackage{color}

%New colors defined below
\definecolor{codegreen}{rgb}{0,0.6,0}
\definecolor{codegray}{rgb}{0.5,0.5,0.5}
\definecolor{codepurple}{rgb}{0.58,0,0.82}
\definecolor{backcolour}{rgb}{0.95,0.95,0.92}

\newcommand{\includecode}[2][c]{\lstinputlisting[caption=#2, escapechar=, style=custom#1]{#2}<!---->}

%Code listing style named "mystyle"
\lstdefinestyle{mystyle}{
  backgroundcolor=\color{backcolour},   commentstyle=\color{codegreen},
  keywordstyle=\color{magenta},
  numberstyle=\tiny\color{codegray},
  stringstyle=\color{codepurple},
  basicstyle=\footnotesize,
  breakatwhitespace=false,         
  breaklines=true,                 
  captionpos=b,                    
  keepspaces=true,                 
  numbers=left,                    
  numbersep=5pt,                  
  showspaces=false,                
  showstringspaces=false,
  showtabs=false,                  
  tabsize=2
}

%"mystyle" code listing set
\lstset{style=mystyle}

\title{Code Listing}
\date{ }

\begin{document}

\maketitle

\section{Code examples}

%Python code highlighting
\begin{lstlisting}[language=Python, caption=Python example]
#!/usr/bin/env python
# -*- coding: utf-8 -*-

# Author  : Bhishan Poudel
# Date    : Mar 26, 2016

# Polynomial : p[0] * x**n + p[1] * x**(n-1) + ... + p[n-1]*x + p[n]
# Coeffs     : p0,p1,p2,...pn

# Program : solve x*3 -50x^2 + 185x - 924.5 = 0

# Imports
import numpy as np

# scriptE = 0
# To solve x^2 - 9.245 x + 18.5 = 0
coeff = [1, -9.245, 18.5]
print("for scriptE = 0")
print(np.roots(coeff))


# scriptE = -0.02
# To solve scriptE * x^3 + x^2 -9.245x + 18.49 = 0
coeff = [-0.03, 1, -9.245, 18.49]
print("\nfor scriptE = -0.02")
print(np.roots(coeff))


# scriptE = 0.03
coeff = [0.03, 1, -9.245, 18.49]
print("\nfor scriptE = 0.03")
print(np.roots(coeff))

# scriptE = 0.025
coeff = [0.025, 1, -9.245, 18.49]
print("\nfor scriptE = 0.025")
print(np.roots(coeff))

# scriptE = 0.045
coeff = [0.045, 1, -9.245, 18.49]
print("\nfor scriptE = 0.045")
print(np.roots(coeff))


#for scriptE = 0
#[ 6.31587127  2.92912873]

#for scriptE = -0.02
#[ 18.41942521  12.16281345   2.75109468]

#for scriptE = 0.03
#[-41.18015683   4.5764082    3.2704153 ]

#for scriptE = 0.025
#[-48.02144629   4.83803695   3.18340934]

#for scriptE = 0.045
#[-29.62518188+0.j           3.70147983+0.41064467j   3.70147983-0.41064467j]

\end{lstlisting}

\clearpage

The next code will be directly imported from a file:

%Importing code from file
\lstinputlisting[language=C, caption=C sample code]{a.c}

\clearpage

\lstlistoflistings

\end{document}
